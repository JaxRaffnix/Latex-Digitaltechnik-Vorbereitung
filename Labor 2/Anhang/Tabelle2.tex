\documentclass [15pt,a4paper,twoside]{article}
\usepackage[german,shorthands=off]{babel}        % shorhands=off is required for babel french in combination with tikz karnaugh....
\usepackage[utf8x]{inputenc}
\usepackage[T1]{fontenc}
\usepackage{amsmath}
\usepackage{geometry}
\geometry{verbose,a4paper, tmargin=3.5cm,bmargin=3.5cm,lmargin=2.5cm,rmargin=2.5cm,headsep=1cm,footskip=1.5cm}
\usepackage{fancyhdr}
\usepackage{colortbl}
\usepackage[dvipsnames]{xcolor}
\usepackage{tikz -timing}
\usepackage{tikz}
\usetikzlibrary{karnaugh}
\pagestyle{fancy}

\definecolor{LogisimKMapColor0}{RGB}{128,0,0}
\definecolor{LogisimKMapColor1}{RGB}{230,25,75}
\definecolor{LogisimKMapColor2}{RGB}{250,190,190}
\definecolor{LogisimKMapColor3}{RGB}{170,110,40}
\definecolor{LogisimKMapColor4}{RGB}{245,130,48}
\definecolor{LogisimKMapColor5}{RGB}{255,215,180}
\definecolor{LogisimKMapColor6}{RGB}{128,128,0}
\definecolor{LogisimKMapColor7}{RGB}{255,255,25}
\definecolor{LogisimKMapColor8}{RGB}{210,245,60}
\definecolor{LogisimKMapColor9}{RGB}{0,0,128}
\definecolor{LogisimKMapColor10}{RGB}{145,30,180}
\definecolor{LogisimKMapColor11}{RGB}{60,180,175}
\definecolor{LogisimKMapColor12}{RGB}{0,130,203}
\definecolor{LogisimKMapColor13}{RGB}{230,190,255}
\definecolor{LogisimKMapColor14}{RGB}{170,255,195}
\definecolor{LogisimKMapColor15}{RGB}{240,50,230}

\fancyhead{}
\fancyhead[C] {Logisim-evolution generiertes Dokument auf Fri Apr 08 23:48:58 CEST 2022}
\fancyfoot[C] {\thepage}
\renewcommand{\headrulewidth}{0.4pt}
\renewcommand{\footrulewidth}{0.4pt}

\makeatother

\begin{document}
\section{Einf�hrung}
Dieses Dokument wurde durch logisim-evolution erstellt. Jeder Teil der TeX-Quellen kann problemlos in Ihren eigenen Dokumenten verwendet werden. Falls Sie alle/Teile dieser generierten TeX-Quellen verwenden m�chten, vergessen Sie bitte (1) nicht, die ben�tigten Pakete anzugeben, und (2) f�gen Sie einen Hinweis hinzu, dass diese Quelle durch logisim-evolution generiert wurde.
%===============================================================================
\section{Wahrheitstabelle}
Die Tabelle kann viel zu gro� sein, um auf der Seite angezeigt zu werden. Zum Zeitpunkt der Generierung wurde keine Berechnung der Gr��e der Tabelle in Bezug auf die Breite / H�he der Seite durchgef�hrt.
%-------------------------------------------------------------------------------
\subsection{Verdichtete Wahrheitstabelle}
\begin{center}
\begin{tabular}{ccc|cc}
$a$&$b$&$ci$&$s$&$c0$\\
\hline
$0$&$0$&$0$&$0$&$0$\\
$0$&$0$&$1$&$1$&$0$\\
$0$&$1$&$0$&$1$&$0$\\
$0$&$1$&$1$&$0$&$1$\\
$1$&$0$&$0$&$1$&$0$\\
$1$&$0$&$1$&$0$&$1$\\
$1$&$1$&$0$&$0$&$1$\\
$1$&$1$&$1$&$1$&$1$\\

\end{tabular}
\end{center}
%-------------------------------------------------------------------------------
\subsection{Vollst�ndige Wahrheitstabelle}
\begin{center}
\begin{tabular}{ccc|cc}
$a$&$b$&$ci$&$s$&$c0$\\
\hline
$0$&$0$&$0$&$0$&$0$\\
$0$&$0$&$1$&$1$&$0$\\
$0$&$1$&$0$&$1$&$0$\\
$0$&$1$&$1$&$0$&$1$\\
$1$&$0$&$0$&$1$&$0$\\
$1$&$0$&$1$&$0$&$1$\\
$1$&$1$&$0$&$0$&$1$\\
$1$&$1$&$1$&$1$&$1$\\

\end{tabular}
\end{center}
%===============================================================================
\section{Karnaugh Diagramme}
Dieser Abschnitt zeigt verschiedene Versionen der Karnaugh-Diagramme der angegebenen Funktionen.
%-------------------------------------------------------------------------------
\subsection{Leere Karnaugh-Diagramme}
\begin{center}
\begin{tikzpicture}[karnaugh,disable bars,x=1\kmunitlength,y=1\kmunitlength,kmbar left sep=1\kmunitlength,grp/.style n args={4}{#1,fill=#1!30,minimum width= #2\kmunitlength,minimum height=#3\kmunitlength,rounded corners=0.2\kmunitlength,fill opacity=0.6,rectangle,draw}]
\karnaughmap{3}{$s$}{{$b$}{$a$}{$ci$}}{}{
\draw[kmbox] (-0.5,2.5)
   node[below left]{$a$}
   node[above right]{$b$, $ci$} +(-0.2,0.2)
   node[above left]{$s$};\draw (0,2) -- (-0.7,2.7);
\foreach \x/\1 in %
{0/00,1/01,2/11,3/10} {
   \node at (\x+0.5,2.2) {\1};
}
\foreach \y/\1 in %
{0/0,1/1} {
   \node at (-0.4,-0.5-\y+2) {\1};
}
}
\end{tikzpicture}
\end{center}
\begin{center}
\begin{tikzpicture}[karnaugh,disable bars,x=1\kmunitlength,y=1\kmunitlength,kmbar left sep=1\kmunitlength,grp/.style n args={4}{#1,fill=#1!30,minimum width= #2\kmunitlength,minimum height=#3\kmunitlength,rounded corners=0.2\kmunitlength,fill opacity=0.6,rectangle,draw}]
\karnaughmap{3}{$c0$}{{$b$}{$a$}{$ci$}}{}{
\draw[kmbox] (-0.5,2.5)
   node[below left]{$a$}
   node[above right]{$b$, $ci$} +(-0.2,0.2)
   node[above left]{$c0$};\draw (0,2) -- (-0.7,2.7);
\foreach \x/\1 in %
{0/00,1/01,2/11,3/10} {
   \node at (\x+0.5,2.2) {\1};
}
\foreach \y/\1 in %
{0/0,1/1} {
   \node at (-0.4,-0.5-\y+2) {\1};
}
}
\end{tikzpicture}
\end{center}
%-------------------------------------------------------------------------------
\subsection{Ausgef�llt in Karnaugh-Diagrammen}
\begin{center}
\begin{tikzpicture}[karnaugh,disable bars,x=1\kmunitlength,y=1\kmunitlength,kmbar left sep=1\kmunitlength,grp/.style n args={4}{#1,fill=#1!30,minimum width= #2\kmunitlength,minimum height=#3\kmunitlength,rounded corners=0.2\kmunitlength,fill opacity=0.6,rectangle,draw}]
\karnaughmap{3}{$s$}{{$b$}{$a$}{$ci$}}
{01101001}{
\draw[kmbox] (-0.5,2.5)
   node[below left]{$a$}
   node[above right]{$b$, $ci$} +(-0.2,0.2)
   node[above left]{$s$};\draw (0,2) -- (-0.7,2.7);
\foreach \x/\1 in %
{0/00,1/01,2/11,3/10} {
   \node at (\x+0.5,2.2) {\1};
}
\foreach \y/\1 in %
{0/0,1/1} {
   \node at (-0.4,-0.5-\y+2) {\1};
}
}
\end{tikzpicture}
\end{center}
\begin{center}
\begin{tikzpicture}[karnaugh,disable bars,x=1\kmunitlength,y=1\kmunitlength,kmbar left sep=1\kmunitlength,grp/.style n args={4}{#1,fill=#1!30,minimum width= #2\kmunitlength,minimum height=#3\kmunitlength,rounded corners=0.2\kmunitlength,fill opacity=0.6,rectangle,draw}]
\karnaughmap{3}{$c0$}{{$b$}{$a$}{$ci$}}
{00010111}{
\draw[kmbox] (-0.5,2.5)
   node[below left]{$a$}
   node[above right]{$b$, $ci$} +(-0.2,0.2)
   node[above left]{$c0$};\draw (0,2) -- (-0.7,2.7);
\foreach \x/\1 in %
{0/00,1/01,2/11,3/10} {
   \node at (\x+0.5,2.2) {\1};
}
\foreach \y/\1 in %
{0/0,1/1} {
   \node at (-0.4,-0.5-\y+2) {\1};
}
}
\end{tikzpicture}
\end{center}
%-------------------------------------------------------------------------------
\subsection{Ausgef�llt in Karnaugh-Diagrammen mit Abdeckungen}
\begin{center}
\begin{tikzpicture}[karnaugh,disable bars,x=1\kmunitlength,y=1\kmunitlength,kmbar left sep=1\kmunitlength,grp/.style n args={4}{#1,fill=#1!30,minimum width= #2\kmunitlength,minimum height=#3\kmunitlength,rounded corners=0.2\kmunitlength,fill opacity=0.6,rectangle,draw}]
\karnaughmap{3}{$s$}{{$b$}{$a$}{$ci$}}
{01101001}{
\draw[kmbox] (-0.5,2.5)
   node[below left]{$a$}
   node[above right]{$b$, $ci$} +(-0.2,0.2)
   node[above left]{$s$};\draw (0,2) -- (-0.7,2.7);
\foreach \x/\1 in %
{0/00,1/01,2/11,3/10} {
   \node at (\x+0.5,2.2) {\1};
}
\foreach \y/\1 in %
{0/0,1/1} {
   \node at (-0.4,-0.5-\y+2) {\1};
}
   \node[grp={LogisimKMapColor1}{0.8}{0.8}](n0) at(1.5,1.5) {};
   \node[grp={LogisimKMapColor2}{0.8}{0.8}](n1) at(3.5,1.5) {};
   \node[grp={LogisimKMapColor3}{0.8}{0.8}](n2) at(0.5,0.5) {};
   \node[grp={LogisimKMapColor4}{0.8}{0.8}](n3) at(2.5,0.5) {};
}
\end{tikzpicture}
\end{center}
\begin{center}
\begin{tikzpicture}[karnaugh,disable bars,x=1\kmunitlength,y=1\kmunitlength,kmbar left sep=1\kmunitlength,grp/.style n args={4}{#1,fill=#1!30,minimum width= #2\kmunitlength,minimum height=#3\kmunitlength,rounded corners=0.2\kmunitlength,fill opacity=0.6,rectangle,draw}]
\karnaughmap{3}{$c0$}{{$b$}{$a$}{$ci$}}
{00010111}{
\draw[kmbox] (-0.5,2.5)
   node[below left]{$a$}
   node[above right]{$b$, $ci$} +(-0.2,0.2)
   node[above left]{$c0$};\draw (0,2) -- (-0.7,2.7);
\foreach \x/\1 in %
{0/00,1/01,2/11,3/10} {
   \node at (\x+0.5,2.2) {\1};
}
\foreach \y/\1 in %
{0/0,1/1} {
   \node at (-0.4,-0.5-\y+2) {\1};
}
   \node[grp={LogisimKMapColor1}{0.8}{1.8}](n0) at(2.5,1) {};
   \node[grp={LogisimKMapColor2}{1.8}{0.8}](n1) at(2,0.5) {};
   \node[grp={LogisimKMapColor3}{1.8}{0.8}](n2) at(3,0.5) {};
}
\end{tikzpicture}
\end{center}
%===============================================================================
\section{Minimale Ausdr�cke}
$s =  \overline{a}  \cdot  \overline{b}  \cdot ci+ \overline{a}  \cdot b \cdot  \overline{ci} +a \cdot  \overline{b}  \cdot  \overline{ci} +a \cdot b \cdot ci$~\\
$c0 = b \cdot ci+a \cdot ci+a \cdot b$~\\
\end{document}
