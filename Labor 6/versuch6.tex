\documentclass[
    paper=a4,
    % fontsize=12,
    % DIV=calc,
    parskip=half,
]{scrreprt}


\usepackage[T1]{fontenc}	
\usepackage[utf8]{inputenc}
\usepackage[ngerman]{babel}
\usepackage{microtype}
\usepackage{libertine}
\usepackage[libertine]{newtxmath} 

\usepackage{xcolor}
\usepackage{booktabs}
\usepackage{graphicx}
\usepackage{subcaption}
\usepackage{pdfpages}
\usepackage{svg}
\usepackage[
    locale		=	DE, 					%	Deutsche Normen
	detect-all,								%	Richtige Font im Textmodus/Mathemodus
	per-mode	=	fraction,				%	Bruchstrich darstellen
	range-units	=	repeat,					%	Einheiten wiederholt darstellen bei sirange
	range-phrase=	{{ bis }},	
]{siunitx}
\usepackage{amsmath}
\usepackage{circuitikz}

\usepackage{hyperref}

\renewcommand*{\familydefault}{\sfdefault}


\definecolor{colordarkblue}{HTML}{4C1DCC}			% color for internal links
\definecolor{colorblue}{HTML}{0480CC}				% color for weblinks
\definecolor{colorgreen}{HTML}{26CC1B}				% color for citations
\definecolor{coloryellow}{HTML}{F0B707}				% color for missing macro
\definecolor{colorred}{HTML}{CC1204}				% color for edit macro
\colorlet{colorgray}{black!40}						% color for editedit macro


\KOMAoptions{
    numbers=noendperiod,
}

\hypersetup{
    bookmarksnumbered   =   true,
    breaklinks          =   true,
    colorlinks          =   true,
    linkcolor           =   colordarkblue,
    urlcolor            =   colorblue,
    citecolor           =   colorgreen,
    pdftitle            =   {Labor 4 Jan Hoegen},
    pdfsubject          =   {Versuchsvorbereitung Labor Digitaltechnik},    
    pdfauthor           =   {Von Jan Hoegen},
}

\ctikzset{
    logic ports=european,
    logic ports/scale=1.0,
    logic ports/fill=lightgray,
    tripoles/european not symbol=ieee circle,
}
\usetikzlibrary{babel}

\newcommand{\tikzmark}[1]{\tikz[overlay,remember picture] \node (#1) {};}
\newcommand{\DrawBox}[3][]{%
    \tikz[overlay,remember picture]{
    \draw[black,#1]
      ($(#2)+(-0.5em,2.0ex)$) rectangle
      ($(#3)+(0.75em,-0.75ex)$);}
}

\graphicspath{/Anhang}

\newcommand{\shadowsection}[1]{%
	\refstepcounter{section}
	\addcontentsline{toc}{section}{\protect\numberline{\thesection}{#1}}
}

\newcommand{\legend}[1]{\par\footnotesize\textbf{Legende}: #1\par}
\newcommand{\figsource}[1]{\par\footnotesize\textbf{Quelle:} #1\par}

\newcommand{\quoteenv}[1]{\glqq #1\grqq} 

\newcommand{\edit}[1]{\textcolor{colorred}{#1}}

\newcommand{\missing}{%
    \textcolor{coloryellow}{MISSING}%
	\PackageWarning{\jobname}{You used the 'missing' macro at this line. Remove it before finalising document.}%
}


\titlehead{
    \textsc{Hochschule Karlsruhe}\\
    University of Applied Sciences\\
    Fakultät für Elektro- und Informationstechnik
    % Studiengang EITB SS 22
    }
\subject{Studiengang Elektro- und Informationstechnik (Bachelor)}
\title{Bericht Digitaltechnik}
\subtitle{Versuch 6: It all adds up now!}
\author{von Jan Hoegen\thanks{Matrikelnumer: 82358. E-Mail: \href{mailto:jan.hoegen@web.de}{jan.hoegen@web.de}}}
\publishers{Betreuer: Prof. Dr.\,-Ing. Jan Bauer}
\date{erstellt am \today}


\begin{document}

\maketitle

\tableofcontents

% \newpage

\chapter{Einleitung und Zielsetzung}   
    Digitale Addition
    Xilinx Vivaldo
    Vergleich zu Versuch 2. Deutlich weniger steckaufwand, keien 6 IC mehr nötig und dennoch doppelt so viele bitstellen.

    4 bit addiernetz in vhdl beschrieben, real mit fpga testen.

\chapter{Hardwareentwurf}

    \section{Funktionsbeschreibung}
    abbildung addierernetz
    kein carry in
    carry out ist MSB

    Mit 4 bit DIP Schalter Zahlen einstellen
    btn0: dip wert für b speichern
    btn1: dip wert für a speichern

    ausgangszustand mit leds darstellen
    4 bits der summe s auf dem breadboard
    carry out über LED1 auf dem Board
    
    \section{Entwurf auf dem Breadboard}
        top level system abbildung
        adder 4 bit muss entwickelt werden
        latch 4 bit speichert den summanden. bereits vorgegeben

        pin zuweisung

\chapter{VHDL-Umsetzung}

    \section{Hardwarebeschreibung}
        top level system Entwurf aus der laboranleitung

        \subsection{halbaddierer}
            schaltbild halbaddierer

            vhdl entitiy halbaddierer

        \subsection{Volladdierer}
            Schaltbild Volladdierer

            vhdl entiity Volladdierer

        \subsection{Addiernetz}
            referenz abbildung 4 bit addiernetz

            vhdl entity addiernetz

    \section{Siumulationsergebnis}
        testmuster halbaddierer, testmuster volladdierer

        bewertung

    \section{Syntheseergebnis}
        constrains file

\chapter{Aufbau der Schaltung}
    foto der schaltung auf dem breadboard

\chapter{Analyseergebnis der Hardware}
    video testen der schlatung mit testmuster aus 4bit adder

\chapter{Schlussfolgerungen und Fazit}

\chapter{Anhang}
    vhdl entity latch 4 bit

\end{document}
